% Beamer Presentation
% LaTeX Template
% Version 1.0 (10/11/12)
%
% This template has been downloaded from:
% http://www.LaTeXTemplates.com
%
% License:
% CC BY-NC-SA 3.0 (http://creativecommons.org/licenses/by-nc-sa/3.0/)
%
%%%%%%%%%%%%%%%%%%%%%%%%%%%%%%%%%%%%%%%%%

%----------------------------------------------------------------------------------------
%   PACKAGES AND THEMES
%----------------------------------------------------------------------------------------

\documentclass{beamer}
\usepackage[utf8x]{inputenc}
\mode<presentation> {

% The Beamer class comes with a number of default slide themes
% which change the colors and layouts of slides. Below this is a list
% of all the themes, uncomment each in turn to see what they look like.

%\usetheme{default}
%\usetheme{AnnArbor}
%\usetheme{Antibes}
%\usetheme{Bergen}
%\usetheme{Berkeley}
%\usetheme{Berlin}
%\usetheme{Boadilla}
%\usetheme{CambridgeUS}
%\usetheme{Copenhagen}
%\usetheme{Darmstadt}
%\usetheme{Dresden}
%\usetheme{Frankfurt}
%\usetheme{Goettingen}
%\usetheme{Hannover}
%\usetheme{Ilmenau}
%\usetheme{JuanLesPins}
%\usetheme{Luebeck}
\usetheme{Madrid}
%\usetheme{Malmoe}
%\usetheme{Marburg}
%\usetheme{Montpellier}
%\usetheme{PaloAlto}
%\usetheme{Pittsburgh}
%\usetheme{Rochester}
%\usetheme{Singapore}
%\usetheme{Szeged}
%\usetheme{Warsaw}

% As well as themes, the Beamer class has a number of color themes
% for any slide theme. Uncomment each of these in turn to see how it
% changes the colors of your current slide theme.

%\usecolortheme{albatross}
%\usecolortheme{beaver}
%\usecolortheme{beetle}
%\usecolortheme{crane}
%\usecolortheme{dolphin}
%\usecolortheme{dove}
%\usecolortheme{fly}
%\usecolortheme{lily}
%\usecolortheme{orchid}
%\usecolortheme{rose}
%\usecolortheme{seagull}
%\usecolortheme{seahorse}
%\usecolortheme{whale}
%\usecolortheme{wolverine}

%\setbeamertemplate{footline} % To remove the footer line in all slides uncomment this line
%\setbeamertemplate{footline}[page number] % To replace the footer line in all slides with a simple slide count uncomment this line

%\setbeamertemplate{navigation symbols}{} % To remove the navigation symbols from the bottom of all slides uncomment this line
}

\usepackage{graphicx} % Allows including images
\usepackage{booktabs} % Allows the use of \toprule, \midrule and \bottomrule in tables

\usepackage{algorithm,algpseudocode}

%----------------------------------------------------------------------------------------
%   TITLE PAGE
%----------------------------------------------------------------------------------------

\title[L'algoritmo di Edmonds-Karp]{L'algoritmo di Edmonds-Karp } % The short title appears at the bottom of every slide, the full title is only on the title page

\author{\\Luca Foschiani} % Your name
\institute[] % Your institution as it will appear on the bottom of every slide, may be shorthand to save space
{
Università degli studi di Udine \\ % Your institution for the title page
\medskip
Advanced Algorithms
}
\date{} % Date, can be changed to a custom date

\begin{document}

\begin{frame}
\titlepage % Print the title page as the first slide
\end{frame}

\begin{frame}
\frametitle{Overview} % Table of contents slide, comment this block out to remove it
\tableofcontents % Throughout your presentation, if you choose to use \section{} and \subsection{} commands, these will automatically be printed on this slide as an overview of your presentation
\end{frame}

%----------------------------------------------------------------------------------------
%   PRESENTATION SLIDES
%----------------------------------------------------------------------------------------

%------------------------------------------------
\section{Il problema del massimo flusso} % Sections can be created in order to organize your presentation into discrete blocks, all sections and subsections are automatically printed in the table of contents as an overview of the talk
%------------------------------------------------

%\subsection{Subsection Example} % A subsection can be created just before a set of slides with a common theme to further break down your presentation into chunks

\begin{frame}
\frametitle{Definizioni: rete di flusso}
Una \textbf{rete di flusso} $G=(V,E)$ è un grafo orientato nel quale ad ogni arco $(u,v)\in E$ è assegnata una capacità non negativa $c(u,v)\geq 0$.\\
Inoltre, assumiamo che se esiste un arco $(u,v)\in E$, allora $(v,u)\notin E$.\\
Individuiamo due nodi $s$ (\textbf{sorgente}) e $t$ (\textbf{pozzo}). Per questi due nodi, abbiamo che per ogni nodo in $V$ esiste almeno un cammino che lo contiene e che va da $s$ a $t$ (ogni nodo è raggiungibile da $s$ e raggiunge $t$).
\end{frame}

\begin{frame}
\frametitle{Definizioni: flusso}
Un \textbf{flusso} in $G$ è una funzione $f:V\times V\rightarrow \mathbb{R}$ che soddisfa le seguenti proprietà:
\begin{itemize}
\item Per ogni $u,v\in V$ richiediamo $0\leq f(u,v)\leq c(u,v)$ (vincolo di capacità)
\item Per ogni $u\in V\setminus\{s,t\}$ richiediamo $\sum\limits_{v\in V}f(v,u)=\sum\limits_{v\in V}f(u,v)$ (conservazione del flusso)
\end{itemize}
La quantità $f(u,v)$ viene chiamata flusso dal nodo $u$ al nodo $v$.\\
Il \textbf{valore} $|f|$ di un flusso $f$ è definito nel seguente modo:\\
$$|f|=\sum\limits_{v\in V}f(s,v)-\sum\limits_{v\in V}f(v,s)$$
Obiettivo del \textbf{problema del massimo flusso} è, data una rete di flusso, trovare il flusso di valore massimo.
\end{frame}

\section{Il metodo di Ford-Fulkerson}

\begin{frame}
\frametitle{Il metodo di Ford-Fulkerson}
Concetti:
\begin{itemize}
\item Rete residua
\item Cammino aumentante
\item Taglio
\end{itemize}
Per poi arrivare al \textbf{Teorema del flusso massimo e taglio minimo} (permette di dimostrare che l'algoritmo trova sempre il flusso massimo).
\end{frame}

\subsection{Rete residua}

\begin{frame}
\frametitle{L'algoritmo di Ford-Fulkerson\\Rete residua}
Dati una rete $G$ e un flusso $f$, la rete residua $G_f=(V_f,E_f)$ permette di identificare i cammini lungo i quali è possibile aumentare il flusso.\\
Ponendo $G=(V,E)$, $s$ sorgente e $t$ pozzo, Possiamo definire la \textbf{capacità residua} $c_f$ nel seguente modo:\\
$$c_f(u,v) =
\left\{
	\begin{array}{ll}
		c(u,v)-f(u,v)  & \mbox{se } (u,v)\in E \\
		f(v,u) & \mbox{se } (v,u)\in E \\
		0 & \mbox{altrimenti}
	\end{array}
\right.$$\\
Il primo caso della definizione corrisponde alla ``capacità residua" degli archi presenti in $G$.\\
Il secondo caso permette all'algoritmo di ``vedere" le quantità di flusso già assegnate agli archi, dando la possibilità di diminuire il flusso su un arco (assegnando al corrispondente arco in $G_f$ un flusso non nullo).\\
In $G_f$ avrò gli stessi nodi presenti in $G$, mentre gli archi saranno tutti gli $(u,v)$ tali che $c_f(u,v)>0$.
\end{frame}

\begin{frame}
\frametitle{Il metodo di Ford-Fulkerson\\Rete residua}
Idea: un flusso individuato nella rete residua permette di aumentare il flusso nella rete originale.\\
Dati $f$ flusso in $G$ e $f'$ flusso in $G_f$, definisco la funzione $(f\uparrow f'):V\times V\rightarrow \mathbb{R}$ nel seguente modo:\\
$$(f\uparrow f')(u,v) =
\left\{
	\begin{array}{ll}
		f(u,v)+f'(u,v)-f'(v,u) & \mbox{se } (u,v)\in E \\
		0 & \mbox{altrimenti}
	\end{array}
\right.$$\\
\begin{block}{Lemma}
La funzione $f\uparrow f'$ è un flusso in $G$ avente valore $|f\uparrow f'|=|f|+|f'|$.
\end{block}
\end{frame}

\begin{frame}
\frametitle{Il metodo di Ford-Fulkerson\\Esempio rete residua}
Rete di flusso (flusso\textbf{/}capacità) e corrispondente rete residua:
\begin{figure}
\includegraphics[width=0.8\linewidth]{1.png}\\
\includegraphics[width=0.8\linewidth]{2.png}
\end{figure}
\end{frame}

\subsection{Cammino aumentante}

\begin{frame}
\frametitle{Il metodo di Ford-Fulkerson\\Cammino aumentante}
Data una rete di flusso $G$ e un flusso $f$, un \textbf{cammino aumentante} è un cammino semplice che va da $s$ a $t$ nella rete residua $G_f$.\\
Dato un cammino aumentante $p$, chiamiamo \textbf{capacità residua} di $p$ la massima quantità di flusso che è possibile mandare su questo cammino:\\
$c_f(p)=min\{c_f(u,v):(u,v) \mbox{ sta in } p\}$.\\
Definiamo inoltre una funzione $f_p:V\times V\rightarrow \mathbb{R}$ nel seguente modo:\\
$$f_p(u,v) =
\left\{
	\begin{array}{ll}
		c_f(p) & \mbox{se } (u,v) \mbox{ sta in } p\\
		0 & \mbox{altrimenti}
	\end{array}
\right.$$\\
\begin{block}{Lemma}
$f_p$ è un flusso in $G_f$ di valore $|f_p|=c_f(p)>0$.
\end{block}
\begin{block}{Corollario}
$f\uparrow f_p$ è un flusso in $G$ di valore $|f\uparrow f'|=|f|+|f_p|>|f|.$
\end{block}
\end{frame}

\subsection{Taglio}

\begin{frame}
\frametitle{Il metodo di Ford-Fulkerson\\Taglio}
Un \textbf{taglio} $(S,T)$ di una rete di flusso $G=(V,E)$ è una partizione di $V$ in $S$ e $T$ tale che $s\in S$ e $t\in T$.\\
Dato $f$ flusso, esprimiamo con $f(S,T)$ il \textbf{flusso che attraversa il taglio}:
$$f(S,T)=\sum\limits_{u\in S}\sum\limits_{v\in T}f(u,v)-
         \sum\limits_{u\in S}\sum\limits_{v\in T}f(v,u)$$
La \textbf{capacità di un taglio} $(S,T)$ è:
$$c(S,T)=\sum\limits_{u\in S}\sum\limits_{v\in T}c(u,v)$$
Il \textbf{taglio minimo} è il taglio di capacità minima fra i tagli della rete.
\end{frame}

\begin{frame}
\frametitle{Il metodo di Ford-Fulkerson\\Taglio}
\begin{block}{Lemma}
Dati $f$ flusso in $G$ con $s$ sorgente e $t$ pozzo, e (S,T) un qualsiasi taglio di $G$, il flusso attraverso $(S,T)$ è $f(S,T)=|f|$.
\end{block}
\begin{block}{Corollario}
Il valore di qualsiasi flusso $f$ su una rete $G$ è limitato superiormente dalla capacità di un qualsiasi taglio di $G$.
\end{block}
\begin{block}{Teorema del flusso massimo e taglio minimo}
Dati $f$ flusso in una rete $G=(V,E)$ con $s$ sorgente e $t$ pozzo, le seguenti condizioni sono equivalenti:
\begin{itemize}
\item $f$ è un flusso massimo in $G$
\item La rete residua $G_f$ non ammette cammini aumentanti
\item $|f|=c(S,T)$ per qualche taglio $(S,T)$ di $G$
\end{itemize}
\end{block}
\end{frame}

\begin{frame}
\frametitle{Il metodo di Ford-Fulkerson\\Pseudocodice}
\begin{algorithm}[H]
    \caption{Ford-Fulkerson(G,s,t)}%\label{euclid}
    \begin{algorithmic}[1]
        \State \textbf{for} each edge $(u,v)\in E$
        \State \ \ \ \ \ \ $(u,v).f = 0$
        \State \textbf{while} esiste un cammino aumentante $p$ in $G_f$
        \State \ \ \ \ \ \ $c_f(p)=min\ \{c_f(u,v):(u,v)\mbox{ sta in }p\}$
        \State \ \ \ \ \ \ \textbf{for} each edge $(u,v)$ in $p$
        \State \ \ \ \ \ \ \ \ \ \ \ \ \textbf{if} $(u,v)\in E$
        \State \ \ \ \ \ \ \ \ \ \ \ \ \ \ \ \ \ \ $(u,v).f = (u,v).f + c_f(p)$
        \State \ \ \ \ \ \ \ \ \ \ \ \ \textbf{else}
        \State \ \ \ \ \ \ \ \ \ \ \ \ \ \ \ \ \ \ $(v,u).f = (v,u).f - c_f(p)$
    \end{algorithmic}
    \label{alg_1}
\end{algorithm}
\end{frame}

\begin{frame}
\frametitle{Il metodo di Ford-Fulkerson\\Complessità}
Una possibile implementazione del metodo consiste nello scegliere il cammino aumentante alla linea 3 in maniera arbitraria: l'algoritmo che si ottiene termina sempre a patto che le capacità degli archi siano valori razionali.\\
Per il caclolo della complessità, possiamo ridurci al caso in cui le capacità prendono valori interi.\\
Il ciclo \textbf{for} alle linee 1-2 ha complessità $O(|E|)$.\\
Se chiamiamo $f^*$ il flusso massimo, abbiamo che il ciclo \textbf{while} alla linea 3 verrà eseguito al più $|f^*|$ volte (ad ogni iterazione il valore del flusso aumenterà almeno di una unità). Trovare un cammino nella rete residua costa $O(|V|+|E|)=O(|E|)$ con BFS o DFS.\\
La complessità dell'algoritmo è quindi $O(|E|\cdot|f^*|)$. L'algoritmo è pseudopolinomiale.
\end{frame}

\begin{frame}
\frametitle{Il metodo di Ford-Fulkerson\\Caso pessimo}
Le prime tre immagini mostrano il primo passo dell'algoritmo e la successiva selezione del cammino aumentante. L'ultima immagine mostra lo stato al quale si arriva dopo 2000 iterazioni.
\begin{figure}
\includegraphics[width=0.3\linewidth]{11.png}
\includegraphics[width=0.3\linewidth]{12.png}\\\includegraphics[width=0.3\linewidth]{13.png}
\includegraphics[width=0.3\linewidth]{14.png}
\end{figure}
\end{frame}

\section{L'algoritmo di Edmonds-Karp}

\begin{frame}
\frametitle{L'algoritmo di Edmonds-Karp}
L'algoritmo di Edmonds-Karp è un'implementazione del metodo di Ford-Fulkerson che ci permette di risolvere il problema del massimo flusso in tempo $O(|V|\cdot|E|^2)$. L'algoritmo utilizza una ricerca in ampiezza per trovare, ad ogni iterazione, il \textbf{cammino minimo} da $s$ a $t$ nella rete residua.
\begin{block}{Lemma}
Dato $G=(V,E)$ con $s$ sorgente e $t$ pozzo, per tutti i $v\in V\setminus\{s,t\}$ abbiamo che la lunghezza del cammino minimo $\delta_f(s,v)$ fra $s$ e $v$ nella rete residua $G_f$ aumenta monotonicamente durante l'esecuzione dell'algoritmo (in particolare, ad ogni aumento del flusso).
\end{block}
\end{frame}

\begin{frame}
\frametitle{L'algoritmo di Edmonds-Karp}
\begin{block}{Dimostrazione}
Supponiamo, per assurdo, che esista un vertice $v\in V\setminus\{s,t\}$ per il quale, in corrispondenza di un aumento del flusso, la distanza da $s$ si riduca. Chiamiamo $f$ il flusso prima di questo aumento, e $f'$ il flusso risultante. Sia $v$ il nodo tale che $\delta_{f'}(s,v)<\delta_f(s,v)$ e per il quale $\delta_{f'}(s,v)$ è minima. Sia inoltre $p=s\leadsto u\rightarrow v$ un cammino minimo da $s$ a $v$ in $G_{f'}$.\\
Abbiamo $\delta_{f'}(s,u)=\delta_{f'}(s,v)-1$.\\
Per come abbiamo scelto $v$, sappiamo inoltre che $\delta_{f'}(s,u)\geq \delta_f(s,u)$.\\
Se $(u,v)\in E_{f}$ avremmo:\\
$\delta_f(s,v)\ \leq\ \delta_f(s,u)+1$\ \ \ (disuguaglianza triangolare)\\
$\ \ \ \ \ \ \ \ \ \ \ \leq\ \delta_{f'}(s,u)+1$\\
$\ \ \ \ \ \ \ \ \ \ \ =\ \delta_{f'}(s,v)$\\
è una contraddizione, quindi $(u,v)\not\in E_f$.
\end{block}
\end{frame}

\begin{frame}
\frametitle{L'algoritmo di Edmonds-Karp}
\begin{block}{Dimostrazione}
Abbiamo mostrato che $(u,v)\not\in E_f$, sappiamo inoltre che $(u,v)\in E_{f'}$. Questo significa che l'aumento di flusso deve aver aumentato il flusso da $v$ a $u$.\\
L'algoritmo che stiamo analizzando aumenta sempre il flusso sul cammino minimo, quindi il cammino minimo da $s$ a $u$ in $G_f$ ha $(v,u)$ come ultimo arco. Abbiamo quindi:\\
$\delta_f(s,v)\ =\ \delta_f(s,u)-1$\\
$\ \ \ \ \ \ \ \ \ \ \ \leq\ \delta_{f'}(s,u)-1$\\
$\ \ \ \ \ \ \ \ \ \ \ =\ \delta_{f'}(s,v)-2$\\
Questo contraddice la nostra assunzione iniziale, cioè $\delta_{f'}(s,v)<\delta_f(s,v)$. Non esiste quindi un nodo $v$ con queste caratteristiche.
\end{block}
\end{frame}

\begin{frame}
\frametitle{L'algoritmo di Edmonds-Karp}
\begin{block}{Teorema}
Data una rete di flusso $G=(V,E)$, l'algoritmo di Edmonds-Karp aumenta il valore del flusso $O(|V|\cdot|E|)$ volte.
\end{block}
\begin{block}{Dimostrazione}
Un arco $(u,v)$ appartenente ad un cammino aumentante è \textbf{critico} se $c_f(p)=c_f(u,v)$. Quando aumentiamo il flusso, tutti gli archi critici appartenenti al cammino aumentante scompaiono dalla rete residua (inoltre, ogni cammino aumentante ha almeno un arco critico).\\
Quello che mostriamo è che ogni arco può diventare critico al più $|V|/2$ volte.
\end{block}
\end{frame}

\begin{frame}
\frametitle{L'algoritmo di Edmonds-Karp}
\begin{block}{Dimostrazione}
Siano $u,v$ nodi in $V$ tali che $(u,v)\in E$. Quando $(u,v)$ diventa critico per la prima volta, avremo $\delta_f(s,v)=\delta_f(s,u)+1$.\\
Dopo aver aumentato il flusso, l'arco $(u,v)$ scomparirà dalla rete residua.\\
Per far sì che $(u,v)$ ricompaia su un altro cammino aumentante, il flusso da $u$ a $v$ deve diminuire, cosa che succede solamente quando $(v,u)$ compare su un cammino aumentante.\\
Sia $f'$ un flusso tale che $(v,u)$ compare su un cammino aumentante.\\
Abbiamo:\\
$\delta_{f'}(s,u)\ =\ \delta_{f'}(s,v)+1$\\
$\ \ \ \ \ \ \ \ \ \ \ \ \geq\ \delta_{f}(s,v)+1$\ \ \ \ (Lemma)\\
$\ \ \ \ \ \ \ \ \ \ \ \ =\ \delta_{f}(s,u)+2$\\
Questo dimostra che da quando $(u,v)$ diventa critico al successivo momento in cui sarà critico, la distanza da $s$ ad $u$ aumenterà almeno di 2.
\end{block}
\end{frame}

\begin{frame}
\frametitle{L'algoritmo di Edmonds-Karp}
\begin{block}{Dimostrazione}
I nodi intermedi di un cammino minimo da $s$ a $u$ non possono mai contenere $s$, $u$ o $t$. Quindi, la sua distanza (dopo ogni iterazione) sarà al massimo $|V|-2$.\\
Dopo che $(u,v)$ è diventato critico per la prima volta, può ridiventarlo al più altre $(|V|-2)/2=|V|/2-1$ volte, per un totale di $|V|/2$ volte.\\
Visto che il numero di archi che possono comparire in una rete residua è $O(|E|)$, il numero totale di archi critici durante l'esecuzione dell'algoritmo sarà $O(|V|\cdot|E|)$
\end{block}
Come visto precedentemente, possiamo individuare un cammino aumentante utilizzando BFS in tempo $O(|E|)$. La complessità dell'algoritmo è quindi
 $O(|V|\cdot|E|^2)$.
\end{frame}

\begin{frame}
Algoritmi basati su ford-fulkerson di complessità minore (push relabel o dinic)
\end{frame}

\end{document}