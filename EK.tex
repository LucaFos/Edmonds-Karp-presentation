% Beamer Presentation
% LaTeX Template
% Version 1.0 (10/11/12)
%
% This template has been downloaded from:
% http://www.LaTeXTemplates.com
%
% License:
% CC BY-NC-SA 3.0 (http://creativecommons.org/licenses/by-nc-sa/3.0/)
%
%%%%%%%%%%%%%%%%%%%%%%%%%%%%%%%%%%%%%%%%%

%----------------------------------------------------------------------------------------
%   PACKAGES AND THEMES
%----------------------------------------------------------------------------------------

\documentclass{beamer}
\usepackage[utf8x]{inputenc}
\mode<presentation> {

% The Beamer class comes with a number of default slide themes
% which change the colors and layouts of slides. Below this is a list
% of all the themes, uncomment each in turn to see what they look like.

%\usetheme{default}
%\usetheme{AnnArbor}
%\usetheme{Antibes}
%\usetheme{Bergen}
%\usetheme{Berkeley}
%\usetheme{Berlin}
%\usetheme{Boadilla}
%\usetheme{CambridgeUS}
%\usetheme{Copenhagen}
%\usetheme{Darmstadt}
%\usetheme{Dresden}
%\usetheme{Frankfurt}
%\usetheme{Goettingen}
%\usetheme{Hannover}
%\usetheme{Ilmenau}
%\usetheme{JuanLesPins}
%\usetheme{Luebeck}
\usetheme{Madrid}
%\usetheme{Malmoe}
%\usetheme{Marburg}
%\usetheme{Montpellier}
%\usetheme{PaloAlto}
%\usetheme{Pittsburgh}
%\usetheme{Rochester}
%\usetheme{Singapore}
%\usetheme{Szeged}
%\usetheme{Warsaw}

% As well as themes, the Beamer class has a number of color themes
% for any slide theme. Uncomment each of these in turn to see how it
% changes the colors of your current slide theme.

%\usecolortheme{albatross}
%\usecolortheme{beaver}
%\usecolortheme{beetle}
%\usecolortheme{crane}
%\usecolortheme{dolphin}
%\usecolortheme{dove}
%\usecolortheme{fly}
%\usecolortheme{lily}
%\usecolortheme{orchid}
%\usecolortheme{rose}
%\usecolortheme{seagull}
%\usecolortheme{seahorse}
%\usecolortheme{whale}
%\usecolortheme{wolverine}

%\setbeamertemplate{footline} % To remove the footer line in all slides uncomment this line
%\setbeamertemplate{footline}[page number] % To replace the footer line in all slides with a simple slide count uncomment this line

%\setbeamertemplate{navigation symbols}{} % To remove the navigation symbols from the bottom of all slides uncomment this line
}

\usepackage{graphicx} % Allows including images
\usepackage{booktabs} % Allows the use of \toprule, \midrule and \bottomrule in tables

\usepackage{algorithm,algpseudocode}

%----------------------------------------------------------------------------------------
%   TITLE PAGE
%----------------------------------------------------------------------------------------

\title[L'algoritmo di Edmonds-Karp]{L'algoritmo di Edmonds-Karp } % The short title appears at the bottom of every slide, the full title is only on the title page

\author{\\Luca Foschiani} % Your name
\institute[] % Your institution as it will appear on the bottom of every slide, may be shorthand to save space
{
Università degli studi di Udine \\ % Your institution for the title page
\medskip
%\textit{john@smith.com} % Your email address
}
\date{} % Date, can be changed to a custom date

\begin{document}

\begin{frame}
\titlepage % Print the title page as the first slide
\end{frame}

\begin{frame}
\frametitle{Overview} % Table of contents slide, comment this block out to remove it
\tableofcontents % Throughout your presentation, if you choose to use \section{} and \subsection{} commands, these will automatically be printed on this slide as an overview of your presentation
\end{frame}

%----------------------------------------------------------------------------------------
%   PRESENTATION SLIDES
%----------------------------------------------------------------------------------------

%------------------------------------------------
\section{Il problema del massimo flusso} % Sections can be created in order to organize your presentation into discrete blocks, all sections and subsections are automatically printed in the table of contents as an overview of the talk
%------------------------------------------------

%\subsection{Subsection Example} % A subsection can be created just before a set of slides with a common theme to further break down your presentation into chunks

\begin{frame}
\frametitle{Definizioni: rete di flusso}
Una \textbf{rete di flusso} $G=(V,E)$ è un grafo orientato nel quale ad ogni arco $(u,v)\in E$ è assegnata una capacità non negativa $c(u,v)\geq 0$.\\
Inoltre, assumiamo che se esiste un arco $(u,v)\in E$, allora $(v,u)\notin E$.\\
Individuiamo due nodi $s$ (\textbf{sorgente}) e $t$ (\textbf{pozzo}). Per questi due nodi, abbiamo che per ogni nodo in $V$ esiste almeno un cammino che lo contiene e che va da $s$ a $t$ (ogni nodo è raggiungibile da $s$ e raggiunge $t$).
\end{frame}

\begin{frame}
\frametitle{Definizioni: flusso}
Un \textbf{flusso} in $G$ è una funzione $f:V\times V\rightarrow \mathbb{R}$ che soddisfa le seguenti proprietà:
\begin{itemize}
\item Per ogni $u,v\in V$ richiediamo $0\leq f(u,v)\leq c(u,v)$ (vincolo di capacità)
\item Per ogni $u\in V\setminus\{s,t\}$ richiediamo $\sum\limits_{v\in V}f(v,u)=\sum\limits_{v\in V}f(u,v)$ (conservazione del flusso)
\end{itemize}
La quantità $f(u,v)$ viene chiamata flusso dal nodo $u$ al nodo $v$.\\
Il \textbf{valore} $|f|$ di un flusso $f$ è definito nel seguente modo:\\
$$|f|=\sum\limits_{v\in V}f(s,v)-\sum\limits_{v\in V}f(v,s)$$
Obiettivo del \textbf{problema del massimo flusso} è, data una rete di flusso, trovare il flusso di valore massimo.
\end{frame}

\begin{frame}
-più sorgenti/più pozzi?\\
-ammettere archi $(u,v)$ con $(v,u)\in E$?\\
-applicazioni? (es. matching bipartito di cardinalità massima)
\end{frame}

\section{L'algoritmo di Ford-Fulkerson}

\begin{frame}
\frametitle{L'algoritmo di Ford-Fulkerson}
Concetti:
\begin{itemize}
\item Rete residua
\item Cammino aumentante
\item Taglio
\end{itemize}
Per poi arrivare al \textbf{Teorema del flusso massimo e taglio minimo} (permette di dimostrare che l'algoritmo trova sempre il flusso massimo).
\end{frame}

\begin{frame}
\frametitle{L'algoritmo di Ford-Fulkerson\\Rete residua}
Dati una rete $G$ e un flusso $f$, la rete residua $G_f$ rappresenta in che modo è possibile cambiare il flusso in $G$.\\
Ponendo $G=(V,E)$, $s$ sorgente e $t$ pozzo, Possiamo definire la \textbf{capacità residua} $c_f$ nel seguente modo:\\
$$c_f(u,v) =
\left\{
	\begin{array}{ll}
		c(u,v)-f(u,v)  & \mbox{se } (u,v)\in E \\
		f(v,u) & \mbox{se } (v,u)\in E \\
		0 & \mbox{altrimenti}
	\end{array}
\right.$$\\
Il primo caso della definizione corrisponde alla ``capacità residua" degli archi presenti in $G$.\\
Il secondo caso permette all'algoritmo di ``vedere" le quantità di flusso già assegnate agli archi, dando la possibilità di diminuire il flusso su un arco (assegnando al corrispondente arco in $G_f$ un flusso non nullo).\\
In $G_f$ avrò gli stessi nodi presenti in $G$, mentre gli archi saranno tutti gli $(u,v)$ tali che $c_f(u,v)>0$.
\end{frame}

\begin{frame}
\frametitle{L'algoritmo di Ford-Fulkerson\\Rete residua}
Idea: un flusso individuato nella rete residua mi permette di aumentare il flusso nella rete originale.\\
Dati $f$ flusso in $G$ e $f'$ flusso in $G_f$, definisco la funzione $(f\uparrow f'):V\times V\rightarrow \mathbb{R}$ nel seguente modo:\\
$$(f\uparrow f')(u,v) =
\left\{
	\begin{array}{ll}
		f(u,v)+f'(u,v)-f'(v,u) & \mbox{se } (u,v)\in E \\
		0 & \mbox{altrimenti}
	\end{array}
\right.$$\\
\begin{block}{Lemma}
La funzione $f\uparrow f'$ è un flusso in $G$ avente valore $|f\uparrow f'|=|f|+|f'|$.
\end{block}
\end{frame}

\begin{frame}
\frametitle{L'algoritmo di Ford-Fulkerson\\Cammino aumentante}
Data una rete di flusso $G$ e un flusso $f$, un \textbf{cammino aumentante} è un cammino semplice che va da $s$ a $t$ nella rete residua $G_f$.\\
Dato un cammino aumentante $p$, chiamiamo \textbf{capacità residua} di $p$ la massima quantità di flusso che è possibile mandare su questo cammino:\\
$c_f(p)=min\{c_f(u,v):(u,v) \mbox{ sta in } p\}$.\\
Definiamo inoltre una funzione $f_p:V\times V\rightarrow \mathbb{R}$ nel seguente modo:\\
$$f_p(u,v) =
\left\{
	\begin{array}{ll}
		c_f(p) & \mbox{se } (u,v) \mbox{ sta in } p\\
		0 & \mbox{altrimenti}
	\end{array}
\right.$$\\
\begin{block}{Lemma}
$f_p$ è un flusso in $G_f$ di valore $|f_p|=c_f(p)>0$.
\end{block}
\begin{block}{Corollario}
$f\uparrow f_p$ è un flusso in $G$ di valore $|f\uparrow f'|=|f|+|f_p|>|f|.$
\end{block}
\end{frame}

\begin{frame}
\frametitle{L'algoritmo di Ford-Fulkerson\\Taglio}
Un \textbf{taglio} $(S,T)$ di una rete di flusso $G=(V,E)$ è una partizione di $V$ in $S$ e $T$ tale che $s\in S$ e $t\in T$.\\
Dato $f$ flusso, esprimiamo con $f(S,T)$ il \textbf{flusso che attraversa il taglio}:
$$f(S,T)=\sum\limits_{u\in S}\sum\limits_{v\in T}f(u,v)-
         \sum\limits_{u\in S}\sum\limits_{v\in T}f(v,u)$$
La \textbf{capacità di un taglio} $(S,T)$ è:
$$c(S,T)=\sum\limits_{u\in S}\sum\limits_{v\in T}c(u,v)$$
Il \textbf{taglio minimo} è il taglio di capacità minima fra i tagli della rete.
\end{frame}

\begin{frame}
\frametitle{L'algoritmo di Ford-Fulkerson\\Taglio}
\begin{block}{Lemma}
Dati $f$ flusso in $G$ con $s$ sorgente e $t$ pozzo, e (S,T) un qualsiasi taglio di $G$, il flusso attraverso $(S,T)$ è $f(S,T)=|f|$.
\end{block}
\begin{block}{Corollario}
Il valore di qualsiasi flusso $f$ su una rete $G$ è limitato superiormente dalla capacità di un qualsiasi taglio di $G$.
\end{block}
\begin{block}{Teorema del flusso massimo e taglio minimo}
Dati $f$ flusso in una rete $G=(V,E)$ con $s$ sorgente e $t$ pozzo, le seguenti condizioni sono equivalenti:
\begin{itemize}
\item $f$ è un flusso massimo in $G$
\item La rete residua $G_f$ non ammette cammini aumentanti
\item $|f|=c(S,T)$ per qualche taglio $(S,T)$ di $G$
\end{itemize}
\end{block}
\end{frame}

\begin{frame}
\frametitle{L'algoritmo di Ford-Fulkerson\\Pseudocodice}
\begin{algorithm}[H]
    \caption{Ford-Fulkerson(G,s,t)}%\label{euclid}
    \begin{algorithmic}[1]
        \State \textbf{for} each edge $(u,v)\in E$
        \State \ \ \ \ \ \ $(u,v).f = 0$
        \State \textbf{while} esiste un cammino aumentante $p$ in $G_f$
        \State \ \ \ \ \ \ $c_f(p)=min\ \{c_f(u,v):(u,v)\mbox{ sta in }p\}$
        \State \ \ \ \ \ \ \textbf{for} each edge $(u,v)$ in $p$
        \State \ \ \ \ \ \ \ \ \ \ \ \ \textbf{if} $(u,v)\in E$
        \State \ \ \ \ \ \ \ \ \ \ \ \ \ \ \ \ \ \ $(u,v).f = (u,v).f + c_f(p)$
        \State \ \ \ \ \ \ \ \ \ \ \ \ \textbf{else}
        \State \ \ \ \ \ \ \ \ \ \ \ \ \ \ \ \ \ \ $(v,u).f = (v,u).f - c_f(p)$
    \end{algorithmic}
    \label{alg_1}
\end{algorithm}
\end{frame}

\begin{frame}
osservazioni: complessità dell'algoritmo
\end{frame}

\section{L'algoritmo di Edmonds-Karp}

\begin{frame}
\end{frame}

\begin{frame}
\frametitle{Paragraphs of Text}
Sed iaculis dapibus gravida. Morbi sed tortor erat, nec interdum arcu. Sed id lorem lectus. Quisque viverra augue id sem ornare non aliquam nibh tristique. Aenean in ligula nisl. Nulla sed tellus ipsum. Donec vestibulum ligula non lorem vulputate fermentum accumsan neque mollis.\\~\\

Sed diam enim, sagittis nec condimentum sit amet, ullamcorper sit amet libero. Aliquam vel dui orci, a porta odio. Nullam id suscipit ipsum. Aenean lobortis commodo sem, ut commodo leo gravida vitae. Pellentesque vehicula ante iaculis arcu pretium rutrum eget sit amet purus. Integer ornare nulla quis neque ultrices lobortis. Vestibulum ultrices tincidunt libero, quis commodo erat ullamcorper id.
\end{frame}

%------------------------------------------------

\begin{frame}
\frametitle{Bullet Points}
\begin{itemize}
\item Lorem ipsum dolor sit amet, consectetur adipiscing elit
\item Aliquam blandit faucibus nisi, sit amet dapibus enim tempus eu
\item Nulla commodo, erat quis gravida posuere, elit lacus lobortis est, quis porttitor odio mauris at libero
\item Nam cursus est eget velit posuere pellentesque
\item Vestibulum faucibus velit a augue condimentum quis convallis nulla gravida
\end{itemize}
\end{frame}

%------------------------------------------------

\begin{frame}
\frametitle{Blocks of Highlighted Text}
\begin{block}{Block 1}
Lorem ipsum dolor sit amet, consectetur adipiscing elit. Integer lectus nisl, ultricies in feugiat rutrum, porttitor sit amet augue. Aliquam ut tortor mauris. Sed volutpat ante purus, quis accumsan dolor.
\end{block}

\begin{block}{Block 2}
Pellentesque sed tellus purus. Class aptent taciti sociosqu ad litora torquent per conubia nostra, per inceptos himenaeos. Vestibulum quis magna at risus dictum tempor eu vitae velit.
\end{block}

\begin{block}{Block 3}
Suspendisse tincidunt sagittis gravida. Curabitur condimentum, enim sed venenatis rutrum, ipsum neque consectetur orci, sed blandit justo nisi ac lacus.
\end{block}
\end{frame}

%------------------------------------------------

\begin{frame}
\frametitle{Multiple Columns}
\begin{columns}[c] % The "c" option specifies centered vertical alignment while the "t" option is used for top vertical alignment

\column{.45\textwidth} % Left column and width
\textbf{Heading}
\begin{enumerate}
\item Statement
\item Explanation
\item Example
\end{enumerate}

\column{.5\textwidth} % Right column and width
Lorem ipsum dolor sit amet, consectetur adipiscing elit. Integer lectus nisl, ultricies in feugiat rutrum, porttitor sit amet augue. Aliquam ut tortor mauris. Sed volutpat ante purus, quis accumsan dolor.

\end{columns}
\end{frame}

%------------------------------------------------
%\section{Second Section}
%------------------------------------------------

\begin{frame}
\frametitle{Table}
\begin{table}
\begin{tabular}{l l l}
\toprule
\textbf{Treatments} & \textbf{Response 1} & \textbf{Response 2}\\
\midrule
Treatment 1 & 0.0003262 & 0.562 \\
Treatment 2 & 0.0015681 & 0.910 \\
Treatment 3 & 0.0009271 & 0.296 \\
\bottomrule
\end{tabular}
\caption{Table caption}
\end{table}
\end{frame}

%------------------------------------------------

\begin{frame}
\frametitle{Theorem}
\begin{theorem}[Mass--energy equivalence]
$E = mc^2$
\end{theorem}
\end{frame}

%------------------------------------------------

\begin{frame}[fragile] % Need to use the fragile option when verbatim is used in the slide
\frametitle{Verbatim}
\begin{example}[Theorem Slide Code]
\begin{verbatim}
\begin{frame}
\frametitle{Theorem}
\begin{theorem}[Mass--energy equivalence]
$E = mc^2$
\end{theorem}
\end{frame}\end{verbatim}
\end{example}
\end{frame}

%------------------------------------------------

\begin{frame}
\frametitle{Figure}
Uncomment the code on this slide to include your own image from the same directory as the template .TeX file.
%\begin{figure}
%\includegraphics[width=0.8\linewidth]{test}
%\end{figure}
\end{frame}

%------------------------------------------------

\begin{frame}[fragile] % Need to use the fragile option when verbatim is used in the slide
\frametitle{Citation}
An example of the \verb|\cite| command to cite within the presentation:\\~

This statement requires citation \cite{p1}.
\end{frame}

%------------------------------------------------

\begin{frame}
\frametitle{References}
\footnotesize{
\begin{thebibliography}{99} % Beamer does not support BibTeX so references must be inserted manually as below
\bibitem[Smith, 2012]{p1} John Smith (2012)
\newblock Title of the publication
\newblock \emph{Journal Name} 12(3), 45 -- 678.
\end{thebibliography}
}
\end{frame}

%------------------------------------------------

\begin{frame}
\Huge{\centerline{The End}}
\end{frame}

%----------------------------------------------------------------------------------------

\end{document}